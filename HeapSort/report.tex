\documentclass[UTF8]{ctexart}
\usepackage{geometry, CJKutf8}
\geometry{margin=1.5cm, vmargin={0pt,1cm}}
\setlength{\topmargin}{-1cm}
\setlength{\paperheight}{29.7cm}
\setlength{\textheight}{25.3cm}
\setlength{\parindent}{0pt}
% \documentclass{article}
\usepackage{array} % 用于额外的列定义功能
\usepackage[dvipsnames]{xcolor}
% useful packages.
\usepackage{amsfonts}
\usepackage{amsmath}
\usepackage{amssymb}
\usepackage{amsthm}
\usepackage{enumerate}
\usepackage{graphicx}
\usepackage{multicol}
\usepackage{fancyhdr}
\usepackage{layout}
\usepackage{listings}
\usepackage{float, caption}

\lstset{
    basicstyle=\ttfamily, basewidth=0.5em
}

% some common command
\newcommand{\dif}{\mathrm{d}}
\newcommand{\avg}[1]{\left\langle #1 \right\rangle}
\newcommand{\difFrac}[2]{\frac{\dif #1}{\dif #2}}
\newcommand{\pdfFrac}[2]{\frac{\partial #1}{\partial #2}}
\newcommand{\OFL}{\mathrm{OFL}}
\newcommand{\UFL}{\mathrm{UFL}}
\newcommand{\fl}{\mathrm{fl}}
\newcommand{\op}{\odot}
\newcommand{\Eabs}{E_{\mathrm{abs}}}
\newcommand{\Erel}{E_{\mathrm{rel}}}

\begin{document}

\pagestyle{fancy}
\fancyhead{}
\lhead{张子乐, 3230104237}
\chead{数据结构与算法第七次作业}
\rhead{Dec.1st, 2024}

\section{HeapSort的设计思路}
HeapSort主要思路为:

1.BuildHeap函数建立最大堆

2.不断做DeleteMax将当前最大元素交换到序列未排序部分的末尾

3.用Heapify函数维持最大堆的性质:parent >= child

代码如下:

\begin{lstlisting}[
    language=C++,
    basicstyle=\ttfamily,
    breaklines=true,
    commentstyle=\itshape\color{black!50!white}
]
template <typename Comparable>
void HeapSort(vector<Comparable> &arr)
{
    BuildHeap(arr);//构建最大堆
    for(int i = arr.size()-1; i > 0; --i)//不断将当前最大元素交换到数组末尾
        DeleteMax(arr, i);
}
\end{lstlisting}
Heapify是堆排序的关键函数:它判断parent与child的大小,将之中的最大元交换到parent的位置。

如果发生交换,它还将在交换后到下一级继续进行判断。

这里没有使用递归,是担心序列元素过多后可能会出现栈溢出的问题。

代码如下:

\begin{lstlisting}[
    language=C++,
    basicstyle=\ttfamily,
    breaklines=true,
    commentstyle=\itshape\color{black!50!white}
]
template <typename Comparable>
void Heapify( vector<Comparable> &arr,int i,int n )//使parent>child
{
    int parent = i;
    while(true)//递归可能导致栈溢出
    {   
        int theMax=parent; //默认parent是最大元
        int leftchild = 2*theMax + 1;
        int rightchild = 2*theMax + 2;
        //比较parent与child的大小
        if( leftchild < n && arr[ theMax ] < arr[ leftchild ] )
            theMax = leftchild;
        if( rightchild < n && arr[ theMax ] < arr[ rightchild ] )
            theMax = rightchild;
        if(theMax != parent)//parent不是最大的
        {
            swap( arr[ parent ], arr[ theMax ]);//交换最大元到parent的位置
            parent=theMax;//交换后到下一级维护堆
        }
        else
            break;
    }
}
\end{lstlisting}
建立最大堆由BuildHeap函数实现:从最后一个非叶子节点开始不断调用Heapify函数,向上建堆

代码如下:

\begin{lstlisting}[
    language=C++,
    basicstyle=\ttfamily,
    breaklines=true,
    commentstyle=\itshape\color{black!50!white}
]
template <typename Comparable>
void BuildHeap(vector<Comparable> &arr)//建立最大堆
{
    int size=arr.size();
    for(int i = size/2-1;i >= 0; --i)//从最后一个非叶子节点开始,向上建堆
    {
        Heapify(arr, i, size);
    }
}
\end{lstlisting}
DeleteMax函数首先将根节点,即最大元,与序列未排序部分的最后一个元素交换。

然后调用Heapify函数重建堆,维护最大堆的性质。

代码如下:

\begin{lstlisting}[
    language=C++,
    basicstyle=\ttfamily,
    breaklines=true,
    commentstyle=\itshape\color{black!50!white}
]
template <typename Comparable>
void DeleteMax(vector<Comparable> &arr, int i)
{
    swap(arr[0], arr[i]);//将根节点即最大元素与数组末尾的元素交换
    Heapify(arr, 0, i);//重建堆维持最大堆的性质
}
\end{lstlisting}


\section{测试流程}
check函数用于检查排序的正确性,代码如下:
\begin{lstlisting}[
    language=C++,
    basicstyle=\ttfamily,
    breaklines=true,
    commentstyle=\itshape\color{black!50!white}
]
template <typename Comparable>
bool check(vector<Comparable> &arr)//检查排序是否正确
{
    for(size_t i = 0; i < arr.size()-1; ++i)
    {
        if(arr[i] > arr[i+1])
            return false;
    }
    return true;
}
\end{lstlisting}
随后构建了四个不同的序列:

1.SortVector构建了一个从1到N(本次取1000000)的递增有序数列。

代码如下:

\begin{lstlisting}[
    language=C++,
    basicstyle=\ttfamily,
    breaklines=true,
    commentstyle=\itshape\color{black!50!white}
]
void SortedVector(vector<int> &arr, int N)//构造有序序列
{
    for(int i = 1; i <= N; ++i)
        arr.emplace_back(i);//emplace_back函数在容器的末尾就地构造一个新元素
}
\end{lstlisting}

2.RandomVector构建了N个伪随机数序列。

代码如下:

\begin{lstlisting}[
    language=C++,
    basicstyle=\ttfamily,
    breaklines=true,
    commentstyle=\itshape\color{black!50!white}
]
void RandomVector(vector<int> &arr, int N)//构造随机序列
{
    random_device seed;//生成随机种子
    mt19937 rnd(seed());//使用seed作为种子初始化mt19937随机数生成器
    for(int i = 0; i <= N; ++i)
    {
        arr.emplace_back(rnd());
    }
}
\end{lstlisting}

3.ReverseVector构造了从N到1的逆序序列。

代码如下:

\begin{lstlisting}[
    language=C++,
    basicstyle=\ttfamily,
    breaklines=true,
    commentstyle=\itshape\color{black!50!white}
]
void ReverseVector(vector<int> &arr, int N)//构造逆序序列
{
    for(int i=N; i >= 1; --i)
        arr.emplace_back(i);
}
\end{lstlisting}

4.RepeatedVector构造了有重复元素的序列并打乱:对于每个正整数i序列中共存在i个

代码如下:

\begin{lstlisting}[
    language=C++,
    basicstyle=\ttfamily,
    breaklines=true,
    commentstyle=\itshape\color{black!50!white}
]
void RepeatedVector(vector<int> &arr, int N)//构造有重复元素的序列
{
    int count = 0;
    for(int i = 1; count <= N; ++i)
    {
        for(int j=1; j <= i; ++j)
        {
            arr.emplace_back(i);
            ++count;
        }
    }
    random_device seed; //生成随机种子
    mt19937 rnd(seed()); // 以seed作为种子生成随机数
    shuffle(arr.begin(), arr.end(),rnd);//打乱序列
}
\end{lstlisting}
testHeapSort和testSTL函数分别用来测试Heapsort和std::sort\_heap。

两者都调用<chrono>库,用start记录排序开始时间,用end记录结束时间。

调用check函数检查排序的正确性,如果正确输出"Correct!"。

输出排序消耗的时间duration=end-start,单位为ms。

每次测试结束后清空序列。



\section{测试结果}
分别用HeapSort和std::sort\_heap对长度为1000000的随机序列、有序序列、逆序序列、部分元素重复序列进行了
测试。结果如下:



\begin{table}[htbp] 
\centering 
\caption{测试结果} 
\begin{tabular}{|c|c|c|} 
\hline 
   \   & my heapsort time & std::sort\_heap time \\ \hline 
random sequence & 131 ms & 132 ms \\ \hline
ordered sequence & 83 ms & 70 ms \\ \hline
reverse sequence & 79 ms & 77 ms \\ \hline
repetitive sequence & 114 ms & 105 ms \\\hline
\end{tabular}
\end{table}

\section{HeapSort时间复杂度分析}
由于堆的高度是$\log N$,因此每一次调用Heapify函数建最大堆的时间复杂度是$\mathcal{O}(log N)$。
除了堆中最后一个元素外,每个元素都要成为一次根节点元素。
总共需要进行N-1次DeleteMax并重建堆的操作。因此HeapSort的时间复杂度是$\mathcal{O}(N \log N)$。

\section{效率差异原因分析}
在std::sort\_heap中,其调用的pop\_heap对应HeapSort中的DeleteMax。

同样将堆顶元素交换到序列末尾后,
它并未直接进行下沉操作,而是通过std::move()函数先把堆顶空出来,把交换来的元素先放在一边。
通过\_\_adjust7\_heap函数向上提升节点后,再把交换来的元素放在末尾。

最后调用\_\_push\_heap调整堆,其实际上是一个向上的插入排序:通过子节点不断与父节点比大小实现。
新加入的元素只会影响一条路径:叶子结点到根节点的路径,因此只要在这条路径上做插入排序就可以了。

在make\_heap中,它同样从最后一个非叶子节点向上调整堆,不断调用\_\_adjust\_heap函数实现。

相比HeapSort,std::sort\_heap的方式可能由于减少了比较次数而效率更高。
\end{document}


%%% Local Variables: 
%%% mode: latex
%%% TeX-master: t
%%% End: 
